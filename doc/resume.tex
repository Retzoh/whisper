\documentclass[10pt,a4paper]{article}
\usepackage[utf8]{inputenc}
\usepackage{amsmath}
\usepackage{amsfonts}
\usepackage{amsthm}
\usepackage{amssymb}
\usepackage{graphicx}
\usepackage{hyperref}

\newtheorem{definition}{Definition}

\author{Temigo Aoiki}
\title{Rumor source inference}
\begin{document}
\maketitle
This is a reference document for whisper project.

\tableofcontents

\newpage
\section{Overview}
Research in epidemiology has been focused on the dynamics of epidemics, infection rates, network structure, etc. However, only very recent work studied the source inference of an epidemics.

In this document, we will will work on graphs that are supposed to be given. (Graph -> Source inference) The modelisation part (Real world -> Graph) won't be studied here.

\section{Epidemic models}
The spread of information should first be designed. \footnote{Good deal of informations on the different models used in epidemiology can be found here : 
\url{http://en.wikipedia.org/wiki/Epidemic_model}}
Here, we only present the SIR model and the SI model, which are the main models studied in the rumor source inference literature.

\subsection{The SIR model}
The well-known model SIR model (susceptible-infected-recovered) is the most popular. It was invented in 1927 by Kermack and McKendrick.
Population is partitioned into three compartiments : 
\begin{itemize}
\item $S$ the susceptible individuals (not yet infected),
\item $I$ the infected population,
\item $R$ the individuals who are no more concerned by the disease (either because they recovered and are immune, or because they died).
\end{itemize}
The sum $N = S + I + R$ remains constant (it is the total number of individuals).
Considering e.g. rates of infection and recovery, we can then define several differential equations on these functions.

\subsection{The SI model}
A variant of SIR is the SI model (susceptible-infected), where there is no possible recovery.

\section{Rumor centrality}
Shah and Zaman provided the first systematic study of the problem of finding the source of a rumor in a network. They defined the topological quantity of \textbf{rumor centrality} .

\subsection{General assumptions}
Let $G(V, E)$ be an undirected graph modelling the network. $V$ is the set of vertices, $E$ is the set of edges.We assume there is only one rumor source $v^{*}$. The model used here is the SI model.

We define the virus graph $G_{N}$ which is the subgraph of G induced by the $N$ infected nodes. The actual source will be denoted $v^{*}$ and the source estimation will be $\hat{v}$.

Each node is equally likely to be the source.
Thus, the Maximum Likelihood Estimator is the best estimator, that is :

\[ \hat{v} = \arg \max_{v \in G_N} P(G_{N} | v^{*} = v) \]
The quantity $P(G_{N} | v^{*} = v)$ (the likelihood of a virus graph given a source) will be difficult to compute, but we will find that this can be approached by a combinatorial problem.

\subsection{Regular trees}

We begin with regular trees, that is, trees where every node has the same degree. A virus graph construction can be simply described by a permutation of the $N$ nodes, which isn't unique. However, not all permutations of the nodes can represent a potential virus graph, due to the structure constraint of the virus graph. Those permutations allowing us to reproduce the virus graph are called \textit{permitted permutations}.

$P(G_{N} | v^{*} = v)$ can then be computed exactly : it will be the sum of the probabilities of all permitted permutations. On a regular tree, these probabilities are all equal to $P$\footnote{"This is because of the memoryless property of the virus spreading time between nodes and the constant degree of all nodes." \cite{shah1}}, with

\[ P = \frac{1}{k} \frac{1}{k + (k-2)} ... \frac{1}{k+(N-2)(k-2)}\]

To understand this, consider the source node : it has $k$ free edges. When it joins a neighbour, the total number of free edges become $k+(k-2)$, and so on. This gives us the probability of any $N$ nodes permitted permutation, given any source in $G_N$.

Now, all we have to do is to compute the number of permitted permutations, that is, the number of ways the virus can spread.

\begin{definition}
$R(v,T )$ is the number of permitted permutations of nodes that result in a tree $T$ and begin with node
$v \in T$ .
\end{definition}

As a consequence, the likelihood turns out to be proportional to $R(v,T)$, so we have

\[ \hat{v} = \arg \max_{v \in G_N} P(G_{N} | v^{*} = v) = \arg \max_{v \in G_N} R(v, G_N) \]

Here, $R(v,G_N)$ is called the \textbf{rumor centrality} of node $v$. The node which maximizes it is called the \textbf{rumor center}.

\subsection{Computing rumor centrality}
This is a combinatorial problem.
\begin{definition}
$T^{v}_{v_j}$ is the number of nodes in the subtree
rooted at node $v_j$ , with node $v$ as the source.
\end{definition}

The goal is to count the number of permitted permutations of the $N$ nodes, describing a virus diffusion beginning at source $v$ and compatible with the virus graph $G_N$. The $k$ neighbors of $v$ are $v_1$, ..., $v_k$.

We have $N$ slots free to describe the permutation. The first node has to be the source $v$. Then the nodes whose subtree is rooted at $v_1$ will be placed in the slots, and there are $T^{v}_{v_1}$ such nodes. They can be ordered in $R(v_1,T^{v}_{v_1})$ different ways. There remains $N-1-T^{v}_{v_1}$ nodes. We continue with the second neighbor, and so on. Multiplying all these quantities gives :

\[ 
R(v, G_N) = \binom{N-1}{T^{v}_{v_1}} \binom{N-1-T^{v}_{v_1}}{T^{v}_{v_2}} ... 
\binom{N-1-\sum_{i=1}^{k-1} T^{v}_{v_i}}{T^{v}_{v_k}}
\prod_{i=1}^{k} R(v_i, T^{v}_{v_i})
\]

Playing around with this big expression gives in the end a simple formula for $R(v, G_N)$ :
\[
R(v, G_N) = N! \prod_{u \in G_N} \frac{1}{T^{v}_{u}}
\]
To do this computation, Shah and Zaman proposed a linear-time algorithm in \cite{shah1}.

\subsection{General trees}
To this point, we have defined and computed the rumor centrality for regular trees. In this simple case, the rumor centrality is an exact ML virus source estimator.

In general trees, all permitted permutations may not have the same probability. However, the rumor centrality remains a good estimator. 

\subsection{General graphs}
To apply the precedent properties on trees, we shall consider the spanning tree induced by the virus graph. As this could be difficult to know, a Breadth-First-Search tree is often used instead.
\\

For more details, see \cite{shah1} (early paper) and \cite{shah2} (more mature paper).

\section{Multiple sources}
These papers did only considered the case of a single rumor source. What if there was multiple sources ? Is it still manageable ?

For more details, see \cite{luo1} and \cite{chen1}.

\section{Suspects}
Suppose you have a set of suspects, that is, an a priori knowledge. See \cite{dong1} for this.

\section{Limited observations}
Previous approachs supposed that the graph and the infected nodes could all be observed. In practice, this tends to be difficult, especially if the graph is a big one, such as with social networks. Thus, it would be interesting to develop new approachs needing only a finite and small number of observers.

Pedro Pinto, from EPFL, tried to address this in \cite{pinto}.


See \cite{seo1} and \cite{pinto}. For multiple observations see \cite{dong2}

\section{Related projects}
Some projects might be mentioned here because they are closely related to epidemiology.
\begin{itemize}
\item STEM (Spatiotemporal Epidemiological Modeler ): \url{http://www.eclipse.org/stem/}
\item Epigrass : \url{http://sourceforge.net/projects/epigrass/}
\item NetLogo : \url{http://ccl.northwestern.edu/netlogo/}
\end{itemize}


\newpage
\bibliographystyle{alpha}
\bibliography{resume}

\end{document}